\documentclass[mathserif]{beamer}

\usetheme[secheader]{Madrid}
\setbeamercovered{transparent=50}

\usepackage{tikz}
%\usetikzlibrary{trees}
\usepackage{url}
\usepackage{graphicx}

\usepackage{pygments}
\usepackage{graphicx}

\providecommand{\code}[1]{{\texttt{\scriptsize{#1}}}}
\providecommand{\inputcode}[1]{
  \begin{block}{}
    \scriptsize{\input{#1}}
  \end{block}
}

\title[OO Design]{Object Oriented Design for Physical Simulation}
\author{Joel Berger}
\institute[UIC]{University of Illinois at Chicago}
\date{October 3, 2012}

\begin{document}

\begin{frame}
  \maketitle
\end{frame}

\begin{frame}{Outline}
  \tableofcontents
\end{frame}

\AtBeginSection{
  \begin{frame}{Outline}
    \tableofcontents[currentsection]
  \end{frame}
}

\section{Programming Styles}

\begin{frame}{Programming Styles}
  There are several common programming styles
  \begin{itemize}
    \item Procedural
    \item Functional
    \item Object Oriented
  \end{itemize}
\end{frame}

\begin{frame}{Procedural Programming}
  \begin{block}{Procedural Programming}
    Functions are like tasks, acting on external variables
  \end{block}
  \visible<2->{\inputcode{styles/procedural}}
  \visible<3->{Depends on global variables}
\end{frame}

\begin{frame}{Functional Programming}
  \begin{block}{Functional Programming}
    Functions are like transforms, taking inputs and returning results
  \end{block}
  \visible<2->{\inputcode{styles/functional}}
  \visible<3->{Lots of data to track}
\end{frame}

\begin{frame}{Object Oriented Programming}
  \begin{block}{Object Oriented Programming}
    Objects have internal data, methods (functions) act on inputs and that data
  \end{block}
  \visible<2->{\inputcode{styles/objective}}
\end{frame}

\section{Object Oriented Programming}

\section{Physical Simulations}

\end{document}
